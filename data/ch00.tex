\lecture{使用说明}{lec:introduction}
\section{简介}
% 这一主题的动机
\begin{frame}{简介}{内容简介}
  该主题称为\alert{西北农林科技大学 {\LaTeX} Beamer 主题},其主要目的
  是:
  \begin{itemize}
  \item<1-> 为广大西北农林科技大学师生提供一个简便易用的Beamer主题。
  \item<2-> 建立标准、规范的演示文稿模板,提高我校师生演示文稿制作质
    量。
  \item<3-> 建立标准、规范的演示文稿模板,提高我校师生制作演示文稿效
    率。
  \item<4-> 推广\qtmark{所想即所得}的演示文稿编写模式,让我校广大师生将精力
    注意在文档内容而不是格式上。
  \end{itemize}
\end{frame}
%%%%%%%%%%%%%%%%

\subsection{协议}
% 协议
\begin{frame}{简介}{协议}
  \begin{itemize}
  \item<1-> 该主题中使用的所有logo的版权属于西北农林科技大
    学\href{http://www.nwsuaf.edu.cn}{http://www.nwsuaf.edu.cn}。
  \item<2-> 若演示文稿中的署名为西北农林科技大学,则可以使用这些logo。
  \item<3-> 使用本主题,请遵守 GNU 通用公共协议 v.3 (GPLv3),有关该协议
    详见:
    \href{http://www.gnu.org/licenses/}{http://www.gnu.org/licenses/}。
    在该协议的允许范围内,可以发布和修改本主题中的任何内容,
  \end{itemize}
\end{frame}
%%%%%%%%%%%%%%%%

\section{安装}
% 通用安装
\begin{frame}{安装}{概述}
  本主题包含4个文件:
  \begin{enumerate}
  \item {\tt beamerthemeNWSUAFsidebar.sty}
  \item {\tt beamerinnerthemeNWSUAFsidebar.sty}
  \item {\tt beamerouterthemeNWSUAFsidebar.sty}
  \item {\tt beamercolorthemeNWSUAFsidebar.sty}
  \end{enumerate}
  可以安装为本地使用,也可以安装为全局使用。\pause
  \begin{block}{本地安装}
    将本主题的4个文件拷贝到当前工作文件夹,即可使用该主题。
  \end{block}
\end{frame}

% 通用安装
\begin{frame}{安装}{全局安装}
  \begin{block}{全局安装}
    \begin{itemize}
    \item 如果希望所有用户都能使用这一主题,则应该将该主题安装到本
      地{\LaTeX}目录树中.
    \item 假设这一目录结构为 {\tt <dirstruct>}。\alert{注意}:如果安装
      了其它的宏包,目录中可能有些内容已存在,此时只需要简单合并 {\tt
        <dirstruct>}即可使用该主题。
    \end{itemize}
  \end{block}
\end{frame}

\subsection{GNU/Linux}
% GNU/Linux中的安装
\begin{frame}{安装}{GNU/Linux}  
  \begin{block}{Ubuntu中的TeX Live}
    \begin{enumerate}
    \item 将 {\tt <dirstruct>} 拷贝到本地{\LaTeX}目录树的根目录. 默认是\\
      {\tt \textasciitilde /texmf}\\
      如果根目录不存在,则创建该目录。 符号 {\tt \textasciitilde} 表示
      家目录, 例如:{\tt /home/<username>}
    \item 在终端中运行如下命令\\
      {\tt \$ texhash \textasciitilde /texmf}
    \end{enumerate}
  \end{block}
\end{frame}
%%%%%%%%%%%%%%%%

% Microsoft Windows
\begin{frame}{安装}{Microsoft Windows}
  \begin{block}{Windows中的TeX Live}
    假设使用默认目录(在高级 TeX Live 安装中,可以更改latex目录树的根目
    录)。
    \begin{enumerate}
    \item 将 {\tt <dirstruct>} 拷贝到本地{\LaTeX}目录树的根目录\\
      {\tt \%USERPROFILE\%\textbackslash texmf}\\
      如果不存在,则创建. XP的默认目录是 {\tt \%USERPROFILE\%} 是\\
      {\tt c:\textbackslash Document and
        Settings\textbackslash<username>},\\
      Vista及更高版本是\\
      {\tt c:\textbackslash Users\textbackslash<username>}
    \item 打开 TeX Live 管理器对话框选择 'Actions'中的'Update filename
      database',并执行.
    \end{enumerate}
  \end{block}
\end{frame}
%%%%%%%%%%%%%%%%

\subsection{Mac OS X}
% Mac OS X的安装
\begin{frame}{安装}{Mac OS X}
  \begin{block}{Mac OS X中的 MacTeX}
    将 {\tt <dirstruct>} 拷贝到本地latex目录树的根目录. 默认是\\
    {\tt \textasciitilde /Library/texmf}\\
    如果不存在,则创建. 符号 {\tt \textasciitilde} 表示家目录, 例
    如:{\tt /home/<username>}
  \end{block}
\end{frame}
%%%%%%%%%%%%%%%%

\subsection{宏包依赖}
% 宏包依赖
\begin{frame}{安装}{宏包依赖}
  除需要 Beamer 类外,本主题需要调用两个宏包
  \begin{itemize}
  \item TikZ\footnote{TikZ 是一个绘制图形的杰出宏包.请参
      考\href{http://www.texample.net/tikz/examples/}{在线示
        例} 或
      \href{http://tug.ctan.org/tex-archive/graphics/pgf/base/doc/generic/pgf/pgfmanual.pdf}{pgf
        用户手册}. }
  \item calc
  \end{itemize}
  这些宏包是{\LaTeX}的通用宏包。
\end{frame}
%%%%%%%%%%%%%%%%

\section{用户接口}
\subsection{主题及选项}
% 主题和选项列表
\begin{frame}{用户接口}{加载主题和主题选项}
  \begin{block}{演示文稿主题}
    加载主题只需要输入\\
    {\tt \textbackslash usetheme[<选项>]\{NWSUAFsidebar\}}\\
    与加载其它主题方法一致,本主题会加载内部、外部和颜色主题并且可以传
    递 {\tt <选项>} 参数.
  \end{block}
  \begin{block}{内部主题}
    使用如下命令加载内部主题\\
    {\tt \textbackslash useinnertheme\{NWSUAFsidebar\}}\\
    内部主题无参数.
  \end{block}
\end{frame}
%%%%%%%%%%%%%%%%

% 主题和选项列表
\begin{frame}{用户接口}{加载主题和主题选项}
  \begin{block}{外部主题}
    使用如下命令加载外部主题\\
    {\tt \textbackslash useoutertheme[<选项>]\{NWSUAFsidebar\}}\\
    目前,外部主题的参数有:
    \begin{itemize}
      \scriptsize
    \item {\tt hidetitle}: 隐藏边栏中的短标题
    \item {\tt hideauthor}: 隐藏边栏中的作者缩写
    \item {\tt hideinstitute}: 隐藏边栏底部的单位缩写
    \item {\tt shownavsym}: 显示导航符号
    \item {\tt left} or {\tt right}: 边栏位置 (默认在右边)
    \item {\tt width=<length>}: 边栏宽度 (默认是 2 cm)
      % 宽度指从垂直分割条的右边到slide的右边
    \item {\tt hideothersubsections}: 除了当前section的subsection隐藏其
      它所有 subsections
    \item {\tt hideallsubsections}: 隐藏所有 subsections
    \end{itemize}
    最后4个选项继承于外部sidebar主题.
  \end{block}
\end{frame}
%%%%%%%%%%%%%%%%

% 主题和选项列表
\begin{frame}[allowframebreaks]{用户界面}{加载主题和主题选项}
  \begin{block}{颜色主题}
    使用如下命令载入颜色主题\\
    {\tt \textbackslash usecolortheme[<选项>]\{NWSUAFsidebar\}}\\
    目前,只支持1个选项
    \begin{itemize}
    \item {\tt lightheaderbg}: 使用浅色页眉背景 (目前是白色).
    \end{itemize}
    该选项创建浅色页眉
  \end{block}
  \pause
  \begin{block}{颜色元素 {\tt NWSUAFsidebar}}
    颜色主题定义了新的 beamer 颜色元素 {\tt NWSUAFsidebar} ,它们的前景
    和背景颜色是:
    \begin{itemize}
    \item fg: {\usebeamercolor[fg]{NWSUAFsidebar}淡蓝
        色 (\{RGB\}\{194,193,204\})}
    \item bg: {\usebeamercolor[bg]{NWSUAFsidebar}深蓝
        色 (\{RGB\}\{33,26,82\})}
    \end{itemize}
    可以采用标准的beamer命令的方式使用这些颜色,如:\\
    {\tt \textbackslash usebeamercolor[<fg or
      bg>]\{NWSUAFsidebar\}}. 详情请参考 beamer 用户手册
  \end{block}
\end{frame}
%%%%%%%%%%%%%%%%

\subsection{编译}
% 编译
\begin{frame}{用户界面}{编译}
  \begin{block}{演示文稿的编译}
    本主题需要至少编译 \alert{3} 次,以保证正确处理页码计数器的数字。
  \end{block}
\end{frame}
%%%%%%%%%%%%%%%%

\subsection{主题修改}
% 如何修改主题
{\setbeamercolor{NWSUAFsidebar}{fg=gray!50,bg=gray}
  \setbeamercolor{sidebar}{bg=red!20}
  \setbeamercolor{structure}{fg=red}
  \setbeamercolor{frametitle}{use=structure,fg=structure.fg,bg=red!5}
  \setbeamercolor{normal text}{bg=gray!20}
  \begin{frame}{用户界面}{主题修改}
    \begin{itemize}
    \item<1-> 主题设置了默认的字体、颜色和布局。
    \item<2-> 然而,可以使用beamer类提供的模板系统方便的修改指定的主题
      元素,请参考beamer用户手册。
    \item<3-> 例如,在这一页中,使用如下方式修改了主题元素
      \begin{itemize}
      \item 修改边栏颜色:\\
        {\tt \textbackslash
          setbeamercolor\{NWSUAFsidebar\}\{fg=gray!50,bg=gray\}} {\tt
          \textbackslash setbeamercolor\{sidebar\}\{bg=red!20\}}
      \item 修改结构元素颜色:\\
        {\tt \textbackslash setbeamercolor\{structure\}\{fg=red\}}\\
      \item 修改帧标题文本颜色和背景颜色: {\tt \textbackslash
          setbeamercolor\{frametitle\}\{use=structure,
          fg=structure.fg,bg=red!5\}}
      \item 修改文本背景颜色{\tt \textbackslash setbeamercolor\{normal
          text\}\{bg=gray!20\}}
      \end{itemize}
    \end{itemize}
  \end{frame}}
%%%%%%%%%%%%%%%%

\subsection{波浪背景}
% NWSUAF波浪背景图案
\begin{frame}{用户界面}{波浪背景图案}
  \begin{block}{波浪背景图案}
    \begin{itemize}
    \item<1-> 在这个演示文稿中,标题页和最后一页使用了波浪背景图案。
      也可以在任意一个单独帧中用下述方法添加背景图案使用\\
      {\tt \{\textbackslash NWSUAFwavesbg\\
        \textbackslash begin\{frame\}[<选项>]\{帧标题\}\{帧子标题\}\\
        \ldots\\
        \textbackslash end\{frame\}\}}
    \item<2-> 理想的方法是设计一个新的命令 {\tt NWSUAFwavesbg} 来添加这
      一背景图案,但本主题中还未能实现,期待你的参与!
    \end{itemize}
  \end{block}
\end{frame}
%%%%%%%%%%%%%%%%

\subsection{宽屏支持}
% 宽屏支持
\begin{frame}{用户界面}{宽屏支持}
  \begin{block}{宽屏支持}
    新式投影仪,甚至是现代的电视机已支持如 16:10 或 16:9 模式的宽屏模式.
    Beamer (>= v. 3.10) 支持多种演示文稿的缩放比例。根据Beamer 用户手册
    (v. 3.10)的77页的8.3节的说明,可以使用如下选项设置显示比例\\
    {\tt\textbackslash documentclass[aspectratio=1610]\{beamer\}}\\
    这一命令设置为 16:10. 也可以是 169, 149, 54, 43 (默认).
  \end{block}
\end{frame}
%%%%%%%%%%%%%%%%

%%%%%%%%%%%%%%%%

\section{主题应用}
{\setbeamercolor{NWSUAFsidebar}{fg=gray!50,bg=gray}
 \setbeamercolor{sidebar}{bg=red!20}
 \setbeamercolor{structure}{fg=red}
 \setbeamercolor{frametitle}{use=structure,fg=structure.fg,bg=red!5}
 \setbeamercolor{normal text}{bg=gray!20}
 \subsection{文件夹结构}
\begin{frame}{主题应用}{文件夹结构}
  \begin{itemize}
  \item 根目录
    \begin{itemize}
    \item {\tt data}$\longrightarrow$*.tex,演示文稿latex源代码(分文
      件组织和管理)
    \item {\tt figure}$\longrightarrow$插图文件
    \item {\tt nwsuaflogo}$\longrightarrow$logo图像文件
    \item {\tt setup}
      \begin{itemize}
      \item {\tt packages.tex}$\longrightarrow$宏包加载管理文件
      \item {\tt format.tex}$\longrightarrow$参数配置及自定义命
        令文件
      \end{itemize}
    \item {\tt beamerthemeNWSUAFsidebar.sty}$\longrightarrow$主题文件
    \item {\tt beamerinnerthemeNWSUAFsidebar.sty}$\longrightarrow$内部
      主题文件
    \item {\tt beamerouterthemeNWSUAFsidebar.sty}$\longrightarrow$外部
      主题文件
    \item {\tt beamercolorthemeNWSUAFsidebar.sty}$\longrightarrow$颜色
      主题文件
    \item {\tt main.tex}$\longrightarrow$ 主控文件
    \item {\tt Makefile}$\longrightarrow$ Make文件,{\tt make clean}用
      于清理中间文件
    \end{itemize}
  \end{itemize}
\end{frame}}
%%%%%%%%%%%%%%%%

\section{问题反馈}
\subsection{已知问题}
% 已知问题
\begin{frame}{问题反馈}{已知问题}
  \begin{description}
  \item[重叠脚注] 这是旧版本beamer的问题,升级到最新版,可以解决这一问
    题。请查
    阅
    \href{https://bitbucket.org/rivanvx/beamer/issue/200/width-of-footnote-in-a-sidebar-theme}{
      错误报告}获取更多的细节.
  \end{description}
\end{frame}
%%%%%%%%%%%%%%%%

\subsection{错误, 意见和建议}
% 问题、意见和建议
\begin{frame}{问题反馈}{错误、意见和建议}
  \begin{itemize}
  \item<1-> 本主题中会存在错误,如果发现了错误,请联系我进行改
    进。\alert{再小的问题也是问题}。
  \item<2-> 如果你有好建议和使用体验改善意见,请联系我进行改进。
  \end{itemize}
\end{frame}
%%%%%%%%%%%%%%%%

\subsection{联系方式}
% 联系方式
\begin{frame}{问题反馈}{联系方式}
  请使用以下方式和我联系:
  \begin{center}
    \insertauthor\\
    \href{mailto:nangeng@qq.com}\\
    信息工程学院317\#\\
    西北农林科技大学
  \end{center}
\end{frame}
%%%%%%%%%%%%%%%%

% \tiny
% \scriptsize
% \footnotesize
% \small
% \normalsize
% \large
% \Large
% \LARGE
% \huge
% \Huge

%%% Local Variables: 
%%% mode: latex
%%% TeX-master: "../main.tex"
%%% End: 
