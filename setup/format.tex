\usetheme[
%%% 外部主题选项
%    hidetitle,           % 隐藏边栏中的短标题
%    hideauthor,          % 隐藏边栏中的作者缩写
%    hideinstitute,       % 隐藏边栏底部的单位缩写
    shownavsym,          % 显示导航符号
    width=1.5cm,           % 边栏宽度 (默认是 2 cm)
%    hideothersubsections,% 除了当前section的subsection隐藏其它所有 subsections
%    hideallsubsections,  % 隐藏所有 subsections
    left,               % 边栏位置 (默认在右边)
%%% 颜色主题选项
    %lightheaderbg       % 页眉背景颜色
  ]{NWSUAFsidebar}

% 如果需要更改主题中不同元素的颜色,请取消相应注释并编辑为喜欢的颜色
% 分割条和边栏颜色:
%\setbeamercolor{NWSUAFsidebar}{fg=red!20,bg=red}
%\setbeamercolor{sidebar}{bg=red!20}
% 结构元素颜色:
%\setbeamercolor{structure}{fg=red}
% 帧标题颜色:
%\setbeamercolor{frametitle}{fg=blue!25}
% 正文文本背景色:
%\setbeamercolor{normal text}{bg=gray!10}
% ... 如果需要更改更多的参数,请参考 beamer 用户手册.
% \setbeamertemplate{blocks}[default]

\definecolor{Descitem}{RGB}{0, 0, 139}

\definecolor{StdTitle}{RGB}{26, 33, 141}
\definecolor{StdBody}{RGB}{213,24,0}

\definecolor{AlTitle}{RGB}{255, 190, 190}
\definecolor{AlBody}{RGB}{213,24,0}

\definecolor{ExTitle}{RGB}{201, 217, 217}
\definecolor{ExBody}{RGB}{213,24,0}



\definecolor{bananayellow}{rgb}{1.0, 0.88, 0.21}
\definecolor{arylideyellow}{rgb}{0.91, 0.84, 0.42}
\definecolor{forestgreen}{rgb}{0.13, 0.55, 0.13}
\definecolor{mordantred19}{rgb}{0.68, 0.05, 0.0}
\definecolor{amber}{rgb}{1.0, 0.75, 0.0}
\definecolor{blue}{rgb}{0.0, 0.0, 1.0}
\definecolor{corn}{rgb}{0.98, 0.93, 0.36}

% Standard block
\setbeamercolor{block title}{fg = Descitem, bg = StdTitle!15!white}
\setbeamercolor{block body}{bg = StdBody!5!white}
% Alert block
\setbeamercolor{block title alerted}{fg = Descitem, bg = AlTitle}
\setbeamercolor{block body alerted}{bg = AlBody!5!white}
% Example block
\setbeamercolor{block title example}{bg = ExTitle}
\setbeamercolor{block body example}{bg = ExBody!5!white}

\setbeamerfont{block title}{size=\scriptsize}
\setbeamertemplate{blocks}[rounded][shadow=true]
\setbeamertemplate{section in toc}[sections numbered]
% \setbeamercolor{section number projected}{bg=black,fg=yellow}
% \setbeamercolor{section in toc}{color=black}
% 不需要导向条符号
%\beamertemplatenavigationsymbolsempty
\setbeamertemplate{navigation symbols}{}

% block environment whose can be adjusted
\newenvironment<>{varblock}[2][.9\textwidth]{%
  \setlength{\textwidth}{#1}
  \begin{actionenv}#3%
    \def\insertblocktitle{#2}%
    \par%
    \usebeamertemplate{block begin}}
  {\par%
    \usebeamertemplate{block end}%
  \end{actionenv}}
%% 自定义相关的名称宏命令
%% ==================================================
%% \newcommand{\yourcommand}[参数个数]{内容}
% 西北农林科技大学各单位名称
\newcommand{\nwsuaf}{西北农林科技大学}
\newcommand{\cie}{信息工程学院}
\newcommand{\ca}{农学院}
\newcommand{\cpp}{植物保护学院}
\newcommand{\ch}{园艺学院}
\newcommand{\cast}{动物科技学院}
\newcommand{\cvm}{动物医学院}
\newcommand{\cf}{林学院}
\newcommand{\claa}{风景园林艺术学院}
\newcommand{\cnre}{资源环境学院}
\newcommand{\cwrae}{水利与建筑工程学院}
\newcommand{\cmee}{机械与电子工程学院}
\newcommand{\cfse}{食品科学与工程学院}
\newcommand{\ce}{葡萄酒学院}
\newcommand{\cls}{生命科学学院}
\newcommand{\cs}{理学院}
\newcommand{\ccp}{化学与药学院}
\newcommand{\cem}{经济管理学院}
\newcommand{\cm}{马克思主义学院}
\newcommand{\dfl}{外语系}
\newcommand{\iec}{创新实验学院}
\newcommand{\ci}{国际学院}
\newcommand{\dpe}{体育部}
\newcommand{\cvae}{成人教育}
\newcommand{\iswc}{水土保持研究所}
% 定义引号命令
\newcommand{\qtmark}[1]{``#1''}

%叉号与对号,需要用到pifont宏包
\newcommand{\goodmark}{\textcolor{green!50!black}{\Pisymbol{pzd}{52}}}
\newcommand{\badmark}{\textcolor{red}{\Pisymbol{pzd}{56}}}

% ==================================================
% TiKz绘图设置
% ==================================================

% 插图路径设置
% ==================================================
\graphicspath{{figures/}}%图片所在的目录
% ==================================================

% 为标题页指定一个 logo
\pgfdeclareimage[height=0.5cm]{titlepagelogo}{nwsuaflogo/nwsuaf_logo_new}% 标题页

\titlegraphic{% 标题页底部
  \pgfuseimage{titlepagelogo}
}


%%% Local Variables:
%%% mode: latex
%%% TeX-master: "../main.tex"
%%% End:
